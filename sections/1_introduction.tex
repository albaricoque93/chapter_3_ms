\documentclass[../main.tex]{subfiles}
\begin{document}
\section{Introduction}

The question of how genetic variation is maintained, despite the effects of selection and drift, continues to be central to the study of evolutionary biology \citep{walsh_evolution_2018}. Classical explanations include overdominance (heterozygote advantage) or frequency-dependent selection, but in the modern era of genomic data all patterns of elevated variation than expected under neutrality tends to be categorized broadly as balancing selection, regardless of the evolutionary mechanism \citep{mitchell-olds_which_2007}.  One of the evolutionary mechanisms coined under balancing selection is sexually antagonistc selection (SAS), which ouccrs when the direction of natural selection on traits or loci differs between the sexes \citep{connallon2018environmental}.

\textit{Paragraph about SAS theory and its limitations}

Sexually antagonist selection has been identified as a powerful engine of speciation, and more recently, as a mechanism that can mantain polymorphisms of otherwise dis-advantageous alleles in a population  \citep{connallon_evolutionary_2018, gavrilets2014sexual}. However, the effect of SAS as has been generally studied under strong assumptions such as homogeneous environments and stationary populations \citep{}. Few studies have explored the effect of SAS with more realistic assumptions, such as \citet{connallon_evolutionary_2018} : description of their findings. However, the effect of environmental fluctuations without local adaptation have not been explored in the context of sexual conflict.

\textit{MCT as a way to study the effect of fluctuations in diversity maintenance}

Modern coexistence theory provides a useful conceptual framework that allows for quantifying the contribution of processes that shape ecological communities \citep{Chesson2000, mayfield2010opposing,hillerislambers2012rethinking,, Adleretal2018, petry2018competition}. At its core, modern coexistence theory is built on models of pairwise interactions among competitors, and the coexistence of these competitors depend on the relative importance of stabilizing mechanisms and fitness differences \citep{Chesson2000, chesson2000general, chesson2003quantifying, letten_linking_2017, Adleretal2018, barabas_chessons_2018}. Stabilizing mechanisms cause a species to buffer its own growth when at high density more than it buffers growth of a competitor, and fitness differences describe differences in competitive abilities or growth rates among species \citep{Chesson2000,chesson2003quantifying, levine2009importance}. Many mechanisms can lead to stabilizing mechanisms such as complimentary resource use \citep{tilman1994competition}, differential responses to spatial and temporal environmental variation \citep{chesson1981environmental, angert2009functional}, and species-specific effects of natural enemies \citep{Paine1966, Paine1969, Connell1972,janzen1970herbivores}. Fitness differences reflect differences in resource use and can establish competitive hierarchies among species \citep{godwin2020empiricist}. When stabilizing mechanisms become sufficiently strong to overcome fitness differences, long-term coexistence is possible \citep{Chesson2000}.

\textit{Paragraph that links MCT and evolution}

In this paragrph we introduce the idea that polymorphism is equal to coexistence from a MCT theory perspective. We also introduce fluctuations in population sizes and fitness values as mechanisms that could enhance the role of SAS and how MCT could help distinguish their effect with the partition of growth rates (Ellner)

% directly from slack
\textit{Paragraph about what we do in this paper}


Here we seek to explicitly apply recent theoretical and analytical advances in coexistence theory to the question of how genetic variation is maintained. We aim to quantify the relative importance of different types of fluctuations to overall stable coexistence, or to exclusion of alleles. We extended a conceptualisation of MCT \citep{ellner_expanded_2019} to examine how fluctuations in fitness differences, fluctuations in population sizes, and their interactions can stabilize or hinder coexistence. In particular we examined :

\begin{itemize}
	\item Can fluctuations in population sizes and fitness differences allow sexually antagonistic allleles to coexist when differences in their fitness would typically not allow them to?
	%\item What is the effect of different types of fluctuations in the proportion of parameter space that allows alleles to coexist?
	\item What is the relative contribution of different types of fluctuations that allow each allele to invade ?
\end{itemize}

%H1: Small fluctuations in fitness differences between males and females allow for the coexistence of sexually antagonistic alleles [to my knowledge this is not an aspect of sexual antagonism that has been studied previously and is likely to be biologically relevant]
% based on discussions (Jan 2021) likely won't include this second hyp
%H2: Bottleneck/founder effects (rapid decrease in population size independent from fitness-related traits) allow for the maintenance of sexually antagonistic alleles in a population [this is relevant to invasive species and species of conservation concern]
%H3: Populations with skewed sex ratios are more likely to have coexisting sexually antagonistic alleles than populations with equal sex ratios [operational sex ratios have gone in and out of vogue as being an important signal/cause of sexual selection]

\end{document}
