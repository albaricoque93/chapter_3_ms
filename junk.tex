
Our results showed that both fluctuations in selection and population sizes can substantially increase the expected genetic variability of sexually antagonistic alleles in a population. First, our results showed an increase in the proportion of allelic coexistence in the parameter space compared to classic theoretical expectations. As a baseline, we show in Fig.~\ref{fig:prop}C the outcome of the control simulation, which matches previous findings that without fluctuations, alleles can coexist in only $\approx 0.38$ of the selection parameter space \citep{connallon2018environmental}. Second, our results showed that each type of fluctuation contributed differently to the coexistence of alleles, and their contribution depended on the allele and sex in which invasion occurred.
.

\subsection*{The effect of fluctuations in allelic coexistence}

When only population sizes fluctuated, the average proportion of coexistence increased with the effect size of fluctuations when fluctuations were large enough (Fig.~\ref{fig:prop}A). Fluctuations with small effect sizes  ($\sigma_{g} < 20$) either decreased or matched the average proportion of the parameter space of allelic coexistence compared to the control simulation. Note that an effect size of $\sigma_{g}= 20$ means that fluctuations increased or decreased population sizes approximately 10\% in each timestep. Above this value, as the effect size of fluctuations increased so did the average proportion of the parameter space where alleles could coexist, reaching up to $\approx 0.50$ (Fig.~\ref{fig:prop}A). Importantly, the average proportion of allelic coexistence was highest when fluctuations were negatively correlated (Fig.~\ref{fig:prop}A).


When only selection fluctuated, the average proportion of coexistence increased nonlinearly as the effect size of fluctuations increased (Fig.~\ref{fig:prop}B). Note, however, that simulations with small effect sizes ($\sigma_{w} < 0.2$) yielded identical results as the control simulation (Fig.~\ref{fig:prop}B). Increases in the effect size of fluctuations after this value dramatically increased the average proportion of the parameter space where alleles could coexist, reaching up to $\approx 0.90$ (Fig.~\ref{fig:prop}B). In contrast to fluctuations in population sizes, the effect of fluctuations in selection was the highest when fluctuations were positively correlated (Fig.~\ref{fig:prop}B).

When \textit{both} population sizes and selection fluctuated,  large fluctuations  increased the average proportion of allelic coexistence  as the effect size of fluctuations increased (Fig.~\ref{fig:heat}). These increments were greater in magnitude when $\rho_{g}= -0.75$ and $\rho_{w}= 0.75$ (Fig.~\ref{fig:heat}). Notably, the effects of fluctuations in selection and population sizes were not synergic. Indeed, at high fluctuations ($\sigma_{g} = 70$ or $\sigma_{w} = 0.9$), the average proportion of coexistence was higher when only selection fluctuated compared to when both selection and population sizes were simultaneously fluctuating (Fig.~\ref{fig:heat}).


\subsection*{The relative contribution of fluctuations}

Despite that fluctuations tended to increase the average proportion of coexistence, not all fluctuations contributed positively to allelic coexistence. To illustrate this point, we show in Fig.~\ref{fig:outcomes}A the outcome of one of our simulations  (white square in Fig.~\ref{fig:heat}). In this simulation, the proportion of coexistence was 0.8, more than double the parameter space compared to the control simulation (Fig.~\ref{fig:prop}C). This increase in the proportion of coexistence is because in parts of the parameter space where selection would typically favor the fixation of one of the alleles, fluctuations in population sizes and fitness values allow their coexistence (grey area in Fig.~\ref{fig:prop}C compared to the grey area in Fig.~\ref{fig:outcomes}A). However, note that there are also parts of the parameter space where coexistence is lost compared to the control simulation ( Fig.~\ref{fig:prop}C compared to  Fig.~\ref{fig:outcomes}A).


Simulations in which population sizes fluctuated, yielded both coexistence gains and losses in the selection parameter space (Fig.~\ref{fig:outcomes}A). In contrast, simulations in which only selection fluctuated, resulted in increases of allelic coexistence (Fig. S1 Supporting Information). This was the main reason why the maximum proportion of coexistence was not achieved when both types of fluctuations were operating simultaneously (Fig.~\ref{fig:heat}). Coexistence gains and losses in the parameter space are the result of the sum of relative contributions of each type of fluctuation.  We show in Fig.~\ref{fig:outcomes}B the $\delta$ values that contributed to the coexistence outcomes shown in Fig.~\ref{fig:outcomes}A.

 Our functional decomposition approach showed that the same type of fluctuation can benefit one allele when it is rare while also creating a disadvantage for the invasion of the other allele. For example, $\delta^{N_{m}}$ benefited allele $j$ as an invader but contributed negatively when allele $k$ was the invader (Fig.~\ref{fig:outcomes}B). In contrast, $\delta^ {N_{f}}$ had the opposite effect (Fig.~\ref{fig:outcomes}B). The relative contribution of fluctuations was consintent in the selection parameter space for each allele: each $\delta$ value was either positive and neutral, or negative and neutral, except for $\delta^0$ which could have both negative and positive contributions for both alleles (Fig.~\ref{fig:outcomes}B). Ultimately, allelic coexistence only happened in parts of the parameter space where alleles had a positive invasion growth rate, which was achieved as long as positive contributions of $\delta$ outweighed the negative contributions. As an example, we highlight in (Fig.~\ref{fig:outcomes}B) a point in the parameter space where coexistence is lost compared to the control simulation. This loss was driven mainly by a strong negative contribution of $\delta^0$ to the invasion growth rate of allele $k$.

 The patterns shown in Fig.~\ref{fig:outcomes}B remained when we examined the relative contributions of fluctuations in different invasion scenarios and across replicates (Fig.~\ref{fig:boxes}). Our results showed that the relative contribution of fluctuations in population sizes depended on the sex where invasion occurred. The relative contribution of fluctuations in the male population, $\delta^{N_{m}}$,  was positive for both $j$ and $k$ alleles when the mutation that introduced them to the population happened in males and negative if invasion occurred via females (Fig.~\ref{fig:boxes}A).The opposite pattern was exhibited by  $\delta^{N_{m}}$ (Fig.~\ref{fig:boxes}A). The correlation between fluctuations $\rho_{g}$ determined the effect of  $\delta^{N_{m}N_{f}}$: it contributed positively to both alleles invading via both pathways when fluctuations were negatively correlated, it had a negligible effect when fluctuations were not correlated, and it had a negative effect when fluctuations were positively correlated (Fig.~\ref{fig:boxes}A).

 In contrast, the effect of fluctuations in selection was independent of the sex where invasion occurred. The values of $\delta^{w_{jm}}$ were positive for the $k$ allele,  and negative  for the $j$ allele, regardless of the via of invasion (Fig.~\ref{fig:boxes}B). The opposite pattern was exhibited by $\delta^{w_{kf}}$ (Fig.~\ref{fig:boxes}B). Thus, fluctuations in selection always benefited the allele that selection did not affect. Finally, the correlation between fluctuations determined the effect of $\delta^{w_{jm}w_{kf}}$;  a negative correlation between fluctuations caused $\delta^{w_{jm}w_{kf}}$ to have negative contributions, . no correlation between fluctuations caused $\delta^{w_{jm}w_{kf}}$ to have a negligible effect, and a positive correlation between fluctuations made $\delta^{w_{jm}w_{kf}}$ to have positive contributions to the growth rates of both alleles (Fig.~\ref{fig:boxes}B).



\subsection*{The effect of sex ratios}

As we show in Fig.~\ref{fig:outcomes}, fluctuations in population sizes can in some instances, cause the loss of coexistence in the parameter space. These losses were driven by stochastic changes of the value $\delta^{0}$ can take (Fig.~\ref{fig:outcomes}B). Note that $\delta^{0}$  captured the effect of fluctuations when they were set to their averages. In the control simulation $\delta^{0}$ captured exclusively the effect of selection (Fig.~\ref{fig:ratios}A). Incorporating fluctuations in selection do not fundamentally change the value of $\delta^{0}$ compared to the control simulation (Fig.~\ref{fig:ratios}A). However, fluctuations in the population size of the male and female populations stochastically change the values of $\delta^{0}$ compared to the control simulations (Fig.~\ref{fig:ratios}A). These stochastic contributions were caused by changes in the sex ratios in our simulations which were not captured by the rest  $\delta$ values and explain the losses of coexistence in the parameter space (Fig.~\ref{fig:ratios}A).
