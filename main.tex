\RequirePackage[]{lineno}
\documentclass[12pt]{article}
\usepackage{caption}
\usepackage{times}
\usepackage{setspace}
\usepackage{longtable}
\usepackage{amsmath}
\usepackage{booktabs}
\usepackage{float}
\usepackage{mathpazo}
\usepackage{times}
\usepackage{tikz}
\usepackage{graphicx}
\usepackage[hmargin=2.25cm, vmargin=2cm, headheight=15.5pt]{geometry}
\usepackage{multirow}
\usepackage{tcolorbox}
\usepackage{multicol}
\usepackage{tabularx}
\usepackage{rotating}
\usepackage{pdflscape}
\captionsetup[figure]{font=small}
\captionsetup[table]{font=small}

\usetikzlibrary{arrows,calc}
\geometry{margin=1in}

%\captionsetup{font=doublespacing, size= footnotesize}% Double-spaced float captions
\doublespacing
\DeclareCaptionJustification{double}{\DoubleSpacing}
% Reasonable page setup


\usepackage[]{natbib}
\bibpunct[; ]{(}{)}{;}{a}{,}{;}

% to avoid things being lost to overleaf comment bubbles
\long\def\authornote#1{%
    \leavevmode\unskip\raisebox{-3.5pt}{\rlap{$\scriptstyle\dagger$}}%
    \marginpar{\raggedright\hbadness=10000
        \def\baselinestretch{0.8}\tiny
        \it #1\par}}
\newcommand{\DBS}[1]{\authornote{DBS: #1}}
\newcommand{\ACL}[1]{\authornote{ACL: #1}}

\usepackage{authblk}
\renewcommand\Affilfont{\small}

\newenvironment{abox}[1]{
  \begin{tcolorbox}[float,title=#1, colback=blue!4]
  \fontsize{9}{10}\selectfont
  \begin{multicols}{2}
}{
  \end{multicols}
  \end{tcolorbox}
}


\newenvironment{ecolettcover}{\maketitle}{\clearpage}
\newenvironment{ecolettabstract}{\clearpage\section*{Abstract}}{\clearpage}
\tikzset{
	%Define standard arrow tip
	>=stealth',
	%Define style for different line styles
	help lines/.style={dashed, thick},
	axis/.style={<->},
	important line/.style={thick},
	connection/.style={thick, dotted},
}

%\title{The structural sensitivity of competition models: how model formulation changes our predictions of species coexistence}
\title{Coexistence of alleles: insights of Modern Coexistence Theory into the maintenance of genetic diversity}
\author[1]{Alba Cervantes-Loreto}
\author[1]{Michelle L.\ Maraffini}
\author[1]{Daniel B.\ Stouffer}
\author[1]{Sarah P.\ Flanagan}


\affil[1]{Centre for Integrative Ecology, School of Biological Sciences\\ University of Canterbury, Christchurch 8140, New Zealand}



% Include the date command, but leave its argument blank.
\date{}

%%%%%%%%%%%%%%%%% END OF PREAMBLE %%%%%%%%%%%%%%%%
\let\oldequation\equation
\let\oldendequation\endequation

\renewenvironment{equation}
  {\linenomathNonumbers\oldequation}
  {\oldendequation\endlinenomath}

% \pagestyle{empty}

\begin{document}
\linenumbers
% Double-space the manuscript.
\baselineskip30pt
\maketitle

\begin{ecolettcover}

%\centerline{{\sc Running Title:} The structural sensitivity of competition models}
\begin{center}
\begin{tabular}{ll}
\hline \\

\bf{Words in abstract}         & to be determined \\
\bf{Words in manuscript}       & to be determined\\
\bf{Number of references}      & to be determined  \\
\bf{Number of figures}			& to be determined \\
\bf{Number of tables} 			& 2 \\
\bf{Number of text boxes}		& 0 \\
\bf{Corresponding author}      & Alba Cervantes-Loreto \\
\bf{Phone}                     & +64~369~2880 \\

\bf{Email}                     & alba.cervantesloreto@pg.canterbury.ac.nz \\
                                                                        \\
\hline
\end{tabular}
\end{center}

\maketitle

\end{ecolettcover}
%%%%%%%%%%%%%%%%%%%
\section{Introduction}
The question of how genetic variation is maintained, despite the effects of selection and drift, continues to be central to the study of evolutionary biology \citep{walsh_evolution_2018}. Classical explanations include overdominance (heterozygote advantage) or frequency-dependent selection, but in the modern era of genomic data, all patterns of elevated variation than expected under neutrality tend to be categorized broadly as balancing selection, regardless of the evolutionary mechanism \citep{mitchell-olds_which_2007}.  One of the evolutionary mechanisms coined under balancing selection is sexually antagonistic selection, which occurs when the direction of natural selection on traits or loci differs between the sexes \citep{connallon2018environmental}.

Sexually antagonistic selection has been identified as a powerful engine of speciation that generally prevents more than one allele to be fixed in a population \citep{gavrilets2014sexual}. The effect of sexually antagonistic selection, however, has been generally studied under strong simplifying assumptions such as constant population sizes and homogeneous environments \citep{kidwell1977regions, pamilo1979genic}. Few studies have explored the effect of sexually antagonistic selection on the maintenance of polymorphism with more realistic assumptions, such as \citet{connallon_evolutionary_2018} that found that classical predictions break down when fluctuations in the environment combined with life-history traits allow local adaptations and promote the maintenance of genetic diversity. The effect of environmental fluctuations without local adaptation, however, has not been studied in the context of sexually antagonistic selection.


The contribution of environmental fluctuations to genetic variability remains a debated issue. Classic theoretical models predict that temporal fluctuations in environmental conditions are unlikely to maintain a genetic polymorphism \citep{hedrick1974genetic,hedrick1986genetic}. However, other studies have found that fluctuating selection can maintain genetic variance on sex--linked traits \citep{reinhold2000maintenance}, or in populations where generations overlap \citep{ellner1994role, ellner1996patterns}. Similarly, temporal changes in population sizes have been shown to mitigate the effect of genetic drift in small populations \citep{pemberton1996maintenance}, or populations with a seed bank \citep{nunney2002effective}. Thus, both fluctuations in selection and population sizes could dramatically change the effect of sexually antagonistic selection in the maintenance of genetic diversity.


Importantly, progress requires more than just identifying if fluctuations increase or decrease genetic diversity, but to quantify how exactly they contribute to its' maintenance \citep{ellner2016quantify}. Modern coexistence theory (MCT) provides a powerful conceptual framework to do so \citep{Chesson2000,chesson1994multispecies, barabas_chessons_2018}. Although its core ideas were formalized in an ecological context \citep{chesson1994multispecies,chesson2000general}, this framework provides the necessary tools to examine the relative contributions of fluctuations to diversity maintenance, which can also be applied to evolutionary contexts  \citep{ellner1996patterns,reinhold2000maintenance}. From an ecological perspective, polymorphism is equivalent to the coexistence of species, and the fixation of either one of the alleles is equivalent to competitive exclusion. The coexistence of alleles, thus, can be examined through the same lens as the coexistence of competing species.

Here, we seek to explicitly apply recent theoretical and analytical advances in MCT to the question of how
genetic variation is maintained under sexually antagonistic selection. Specifically, we aim to quantify the relative importance of different types of fluctuations to overall stable coexistence, or to exclusion of sexually antagonistic alleles. We extended a conceptualization of MCT \citep{ellner2016quantify,ellner_expanded_2019} to examine how fluctuations in selection values, fluctuations in population sizes, and their interactions can stabilize or hinder the coexistence of alleles. In particular, we examined i) Can fluctuations in population sizes and selection values allow sexually antagonistic alleles to coexist when differences in their fitness would typically not allow them to? and ii) What is the relative contribution of different types of fluctuations that allow two sexually antagonistic alleles to be maintained in a population? Our study provides the tools to analyze evolutionary dynamics from an ecological perspective and contributes to answering long-lasting questions regarding the effect of non-constant environments on genetic diversity.



\section{Methods}
We first present the evolutionary consequences of sexually antagonistic selection in constant environments. We then present a model that describes the evolutionary dynamics of sexually antagonistc alleles, and show how changes in  in allele's frequencies can be exrpessed in terms of growth rates, a necessary condition for analyses done using MCT. We continue by  simulating different scenarios of alleles invading a population, where we allowed population sizes, selection values, both, or neither to vary. Finally, we examine the results of our simulations through a MCT lense by  calculating the contribution of each of these fluctuations in the coexistence of alleles across the parameter space of sexually antagonist selection.


\subsection*{Sexually antagonistic selection}

 Most population genetic models of sex-dependent selection consider evolution at single, biallelic  loci with frequency and density independent effects on the relative fitness of females and males \citep{wright1942statistical,kidwell1977regions, immler2012ploidally}. Consider a locus with two alleles, $j$ and $k$, that affect fitness in the haploid state.  Assume allele $j$ always has a high fitness in females ($w_{jf} = 1$), but has variable fitness in males ($w_{jm} < 1$); and allele $k$ always has a high fitness in males ($w_{km} = 1$), but has variable fitness in females ($w_{kf} < 1 $). The selection against allele $j$ in males is therefore $S_{m}= 1 - w_{jm}$, and the selection against allele $k$ in females is $S_{f}= 1 - w_{kf}$ . Selection mantains both alleles in the population under the condition that:

 \begin{equation}
\frac{S_{m}}{1+S_{m}} < S_{f} < \frac{S_{m}}{1-S_{m}}
\label{selection}
 \end{equation}
\citep{kidwell1977regions,pamilo1979genic,connallon_evolutionary_2018}. These inequalities can be used to calculate the proportion of the selection parameter space (within the range $ 0 < S_{m}, S_{f} < 1$) that leads to polymorphism of sexually antagonistic alleles: in $\approx 0.31$ of the parameter space allele $j$ will be fixed, in another $\approx 0.31$ of the parameter space allele $k$ will be fixed, and in $\approx 0.38$ of the parameter space polymorphism or coexistence of alleles can be maintained.

Most of the models used to explore the evolutionary dynamics of sexual antagonism assume constant population sizes and homogeneous environments \citep{kidwell1977regions,pamilo1979genic, immler2012ploidally}. In constant environments, the maintenance of polymorphism of sexually antagonistic alleles is solely determined by the values of $S_{m}$ and $S_{f}$. If fluctuations in population sizes or selection values have an effect on the coexistence of sexually antagonistic alleles, it would be reflected in increases or decreases of the proportion of the parameter space of selection where polymorphism is maintained. Furthermore


\subsection*{Population dynamics of sexually antagonistic alleles}
As a baseline, we used a model that captures the effects of sexually antagonistic selection. Our model consisted of a population that has discrete generations, and that is subject to the previously described sexual antagonism between allele $j$ and $k$. The frequency each allele in each sex at the beginning of a life-cycle at time $t$ given by:
\begin{equation}
    p_{jm,t}= \frac{n_{jm,t}}{N_{m,t}}
    \label{first_pop}
\end{equation}
\begin{equation}
    p_{jf,t}= \frac{n_{jf,t}}{N_{f,t}}
\end{equation}
\begin{equation}
    p_{km,t}= \frac{N_{m,t}-n_{jm,t}}{N_{m,t}}
\end{equation}
\begin{equation}
    p_{kf,t}= \frac{N_{f,t}-n_{jf,t}}{N_{f,t}}
\end{equation}
where $N_{m,t}$ and $N_{t,t}$ are the numbers of males and females in a population at time $t$, $n_{jf,t}$ is the number of females $f$ with allele $j$, and $n_{jm,t}$ is the number of males $m$ with allele $j$ at time $t$, respectively.

The individuals in the population mate at random before selection occurs, and therefore the frequency of offspring with allele $j$ after mating, $p'_{j,t}$ can be expressed as:
%Convert frequencies to counts (abundances) to move to coexistence framework. The explicit calculation of $p'_{j}$ prime from random mating what selection acts on yields
\begin{equation}
   p'_{j}= \frac{n_{jf,t}}{N_{f,t}} \frac{n_{jm,t}}{N_{m,t}} + \frac{1}{2} \frac{n_{jf,t}}{N_{f,t}} \frac{(N_{m,t}-n_{jm,t})}{N_{m,t}} +\frac{1}{2}
   \frac{(N_{f,t}-n_{jf,t})}{N_{f,t}} \frac{n_jm,t}{N_{m,t}} \,,
\end{equation}
which upon rearranging and simplifying can be written as:
\begin{equation}
   p'_{j,t}= \frac{(N_{m,t}n_{jf,t}+ N_{f,t}n_{jm,t})}{2 N_{f}N_{m}} \,.
   \label{pprime}
\end{equation}

Selection acts upon these offspring in order to determine the allelic frequencies in females and males in the next generation, $t+1$. As an example the frequency  of females with allele $j$ after selection is given by:
\begin{equation}
   p^{\prime}_{jf, t+1}= \frac{n_{jf, t+1}}{N'_{f,t+1}} = \frac{p'_{j}w_{jf}}{p'_{j}w_{jf}+ (1-p'_{j})w_{kf}}
\end{equation}

The logarithmic growth rate of $j$ in females, is therefore given by the number of females with allele $j$ after selection, divided by the original number of females carrying allele $j$:



\begin{equation}
    r_{jf,t} = \ln \left( \frac{n'_{jf, t+1}}{n_{jf,t}} \right)
    \label{canonical}
\end{equation}
%Which after substitution can be expressed as:

%\begin{equation}
%  r_{jf} = \ln \left[ \frac{N'_{f}}{n_{jf}} \left( \frac{(N_{m}n_{jf}+ N_{f}n_{jm})w_{jf}}{(N_{m}n_{jf}+ N_{f}n_{jm})w_{jf} + (2N_{m}N_{f}-N_{m}n_{jf}-N_{f}n_{jm})w_{kf}} \right) \right] \,.
%\label{usefulequation}
%\end{equation}

An equivalent expression for the per capita growth rate of allele $j$ in males $m$ can be obtained by exchanging $f$ for $m$ across the various subscripts in this expression.

Allelic coexistence in a sexual population, however, is ultimately influenced by growth and establishment of an allele across both sexes. Therefore, the full growth rate of allele $j$ across the entire population of females \emph{and} males is given by
\begin{equation}
    r_{j} = \ln \left( \frac{n'_{jf, t+1} + n'_{jm, t+1} }{n_{jf,t} + n_{jf,t} }  \right) \,.
    \label{full}
\end{equation}

 Equivalently, there exists an expression for $r_{k}$. If the values of selection are within the bounds of Eqn.\ref{selection}, both alleles will have positive growth rates, and therefore be able to coexist. If one allele has a positive growth rate, while the other does not, then only one allele will be fixated into the population.  We used this model as a baseline to perform simulations that allowed us to examine how different types of fluctuation change the expected effect of sexually antagonistic selection.

\subsection*{Simulations}

Although the evolutionary dynamics of sexually antagonistc selection is often explored though changes in alleles' frequencies, MCT requires population dynamics to be expressed as growth rates of the competin alleles, as we show in Eqn.\ref{full}. This is because MCT provides a framework to quantify what gives an allele a population growth rate advantage over the other allele when it becomes rare (i.e., when it is an invader) \citep{chesson_stabilizing_1982, chesson2003quantifying, barabas_chessons_2018}. Our simulations, thus, consisted of performing  invasion simulations of both alleles invading separetly, allowing population sizes and fitness values to fluctuate, across the selection parameter space of sexually antagonistic selection. For simplicity, we first present our approach focusing on a fixed point in the selection parameter space.

%Then, we used MCT to analyze the relative contributions of fluctuations in selection values and population sizes to the coexistence of sexually antagonistic alleles.

\subsubsection*{Timeseries}

%For given values of $w_{jm}$ and $w_{kf}$, we simulated our model (Eqns.~\ref{first_pop} to \ref{full}) 500 timesteps, therefore creating an allele dynamic timeseries. We started each simulation with  initial values of $N_{m}$ and $N_{f}$  of 200 individuals each, and equal frequencies of allele $j$ and allele $k$ in each sex.
We first incorporated the effects of fluctuations into our populatin dynamics model. To do so, we  generated independent timeseries of fluctuations in fitness values and population sizes. In the case of fluctuations in selection values, for a given value of $w_{jm}$ and $w_{kf}$ (i.e., a fixed point in the selection parameter space), we generated a timeseries of 500 timesteps made up of correlated fluctuations of $w_{jm}$ and $w_{kf}$. Following the approach of  \citet{shoemaker2020} we controlled the effect size of  fluctuations in fitness values ($\sigma_{w}$) and its´ correlation ($\rho_{w}$) by  using the Cholesky factorisation of the variance-covariance matrix:

\begin{equation}
C_{w} = \begin{bmatrix}
\sigma_{w}^{2} & \rho_{w} \sigma_{w}^{2} \\
\rho_{w} \sigma_{w}^{2} & \sigma_{w}^{2}
\end{bmatrix}
\label{covmat}
\end{equation}

We multiplyed Eqn.~\ref{covmat} by a ($2 \times 500$) matrix of random numbers from a normal distribution with mean 0 and unit variance, which yielded $\gamma_{j}$ and $\gamma_{k}$. Then, we calculated the value of $w_{jm}$ at time $t+1$ as $w_{jm,t+1} = w_{jm}^{\gamma_{j,t}}$. We calculated the value of $w_{kf,t+1}$  analogously.

Similarly, we generated a timeseries of 500 timesteps made up of correlated fluctuations in population sizes. We chose values of $N_{m}$ and $N_{f}$ of 200 individuals each as the initial value of population sizes througout our simulations. We performed a Cholesky factorisation of the variance-covariance matrix, controlling the effect size of fluctuations in population sizes with $\sigma_{g}$ and their correlation with $\rho_{g}$. We multiplied this factorisation by a random matrix of uncorrelated random variables, which yielded $\gamma_{m}$ and $\gamma_{f}$. Finally, we calculated the number of males in the population at time $t+1$ as $N_{m,t+1} = N_{m,t} + \gamma_{m,t}$. We calculated the value of $N_{f,t+1}$  analogously.

Finally, we performed simulations that allowed our population dynamics model (Eqns.~\ref{first_pop} to \ref{full}) to iterate for 500 timesteps. We started each simulation with the inital values of $N_{m}$ and $N_{f}$ described before and equal frequencies of allele $j$ and allele $k$ in each sex. For each timestep $t$ in our simulations, the values of $w_{jm}$ $w_{kf}$, $N_{m}$ and $N_{f}$ used to calculate allele´s frequencies in the next generation corresponded to the values calculated as described previously. This approach yielded a final timeseries that captured the dynamics of sexually antagonistic alleles, with flucting values of selection and population sizes.

\vspace{5mm}
\noindent\textbf{Invasion simulations}

We used the timeseriesdescribed previously to perform invasion simulations of both alleles. Each allele could invade via two different pathways: males and females. We explored all of the combinations of each allele invading through a different pathway (e.g., allele $j$ invading through males, and allele $k$ invading through females, and so on). Therefore, for every point in the parameter space of sexually antagonistic selection, we explored four different types of invasion.

For each timestep in the timeseries, we performed simulations of the two alleles invading separetly via their respective pathway. To simulate invasion, we set the initial values of the invading allele to one individual, while the resident allele was set to the correspoing value of the timeseries, and we projected forward one generation. For example, if allele $j$ was invading via males, then we would set $n_{jm} = 1$ and $n_{jf}= 0$, while the allele $k$ would be the resident. The abundance of the resident was determined by the timestep $t$ of the timeseries. After one generation, we calculated the logarithmic growth rate of \textit{j} allele invading as:

\begin{equation}
r_{j} =	\ln \left ( \frac{n_{jm,t +1 } + n_{jf,t+1}}{1} \right )
\label{invader}
\end{equation}

Correspondingly, the logarithmic growth rate of the \textit{k} allele as a resident would be given by:
\begin{equation}
r_{k} =	\ln \left ( \frac{ n_{km,t+1} + n_{kf,t+1} }{ n_{km,t} + n_{kf,t}  } \right )
\label{resident}
\end{equation}

We treated each timestep of the timeseries independently, so we performed 500 invasion simulations, one for each timestep. Then, we calculated the mean invasion growth rate as the average of the 500 invasion growth rates, and the mean reasident growth rate as the average of the 500 resident growth rates. We determined alleles to be coexisting if both of them, invading via their respective pathway, had positive  mean invasion growth rates, which is also called the mutual invasibility criterion.

\vspace{5mm}
\noindent\textbf{Functional decompostion}

To understand the relative contribution of fluctuations in population sizes and fitness values, we applied the functional decomposition framework we previously described. To do so, we performed another set of invasion simulations of each allele invading via its corresponding pathway, but setting all of the  fluctuating variables to their means. Then, we calculated invader and resident mean growth rates as previously described (e.g., Eqns.\ref{invader} and \ref{resident}). When every variable was set to its mean, the average invasion and resident growth rate was equal to $\mathcal{E}^{0}$.

Building upon this baseline, we performed another set of invasion simulations, but this time allowing variables to fluctuate one by one, to capture their main effects, and jointly, to capture their interactions. Then, we  caclulated the corresponding values of each $\mathcal{E}$ term, as shown in Table \ref{tab:EllnerRs}. For simplicity, we only show the functional decomposition of $j$ as an invader in Table \ref{tab:EllnerRs}, however, the functional decomposition of $k$ as an invader is identical.  This approach allowed us to capture the contribution of fluctuations to invader and resident growth rates, which we did for each allele invading a different pathway.

Having done the decomposition of invader and resident growth rates, we continued to do the invader-resident comparisons to calculate $\Delta$ values (e.g.,~\ref{delta}). For each allele invading via a different pathway, we calculated 16 $\Delta$ values, one for each one of the $\mathcal{E}$ terms. However, since the magnitude of each one of these values could vary considerably, to make them comparable, we normalized each $\Delta$ value by dividing it by the lenght of the $\boldsymbol{\Delta}$ vector. For example, the normalized value of Eqn.~\ref{delta} would be given by:

\begin{equation}
  \Delta^{Nm*}_{j}= \frac{\Delta^{Nm}_{j}}{\sqrt{
    \sum\limits_{i=1}^{16} (\boldsymbol{\Delta_{i}})^{2} }}
\end{equation}

This normalization bounded $\Delta$ values from $-1$ to $1$.

\vspace{5mm}
\noindent\textbf{The parameter space of sexually antagonist selection}



To evaluate if fluctuations in fitness values and population sizes allow sexually antagonistic alleles to coexist when their fitness values would typically not allow them to, we applied the approach presented so far to the whole parameter space of selection  ($ 0 < S_{m}, S_{f} < 1$). To do so, we partinioned the parameter space in 2500 parts, each one a combination of different $w_{jm}$ and $w_{kf}$ values. For each parameter combination, we  separetly calculated each allele's mean invasion growth rate when invading through males and females, as well as its functional decomposition. Then, we determined coexistence outcomes using the mutual invasibility criterion. Finally, we calculated the proportion of the parameter space that allowed alleles to coexist, for each allele invading via a different sex.

We explored all of  the combinations of low ($\sigma_{w} = 0.1$ and $\sigma_{g}=1$), intermediate (($\sigma_{w} =  0.3$ and $\sigma_{g}=10$)) and high fluctuations ($\sigma_{w} = 0.7$ and $\sigma_{g}=30$) in fitness values and population sizes, with different extents of correlations between fluctuations (Table \ref{tab:fluctuations}).  As a control simulation, we set $\sigma_{w}= 0.001$ and  $\sigma_{g}=0.001$, with no correlation between fluctuations. For each one of the factorial combinations of $\sigma_{g}$, $\sigma_{w}$, $\rho_{g}$ and $\rho_{w}$ (Table \ref{tab:fluctuations}), we performed invasion simulations across the parameter space of selection. We did three replicates per parameter combination, which resulted in 432 simulations.


\clearpage
\subsection*{Results}



\clearpage
\subsection*{Figures and tables }




\begin{table}[h]
\fontsize{7}{12}\selectfont % the spacing on this is rough
    \centering
      \caption{Functional decomposition of the growth rate of allele $j$. Need to get rid of the sums and $m$ because we are only presenting $j$. As well to add an overbar over rj. }
  \resizebox{\textwidth}{!} {\begin{tabular}{l|l|l}
  \toprule
        Term & Formula & Meaning \\
        \hline
         $\mathcal{E}^{0}_{j}$ & $\overline{r_{j}} (\overline{N_{m}}, \overline{N_{f}}, \overline{w_{jm}}, \overline{w_{kf}})$ & Growth rate at mean population size and fitness values. \\


         $\overline{\mathcal{E}}^{N_{m}}_{j}$ & $\overline{r}_{j}(N_{m} \overline{N_{f}}, \overline{w_{jm}}, \overline{w_{kf}}) - \mathcal{E}^{0}_{j} $ & Main effect of fluctuations in $N_{m}$\\

         $\overline{\mathcal{E}}^{N_{f}}_{j}$ & $ \overline{r_{j}}( \overline{N_{m}}, N_{f},\overline{w_{jm}}, \overline{w_{kf}}) - \mathcal{E}^{0}_{j}$ & Main effect of fluctuations in $N_{f}$ \\

        $\overline{\mathcal{E}}^{w_{jm}}_{j}$ & $ \overline{r_{j}}(\overline{N_{m}}, \overline{N_{f}}, w_{jm}, \overline{w_{kf}}) - \mathcal{E}^{0}_{j}$& Main effect of fluctuations in $w_{jm}$\\

        $\overline{\mathcal{E}}^{w_{kf}}_{j}$ & $ \overline{r_{j}}(\overline{N_{m}}, \overline{N_{f}}, \overline{w_{jm}}, w_{kf})- \mathcal{E}^{0}_{j}$ & Main effect of fluctuations in $w_{kf}$\\

        $\overline{\mathcal{E}}^{N_{m},N_{f}}_{j}$ & $ \overline{r_{j}}(N_{m}, N_{f}, \overline{w_{jm}}, \overline{w_{kf}})- [\mathcal{E}^{0}_{j} +\overline{\mathcal{E}}^{N_{m}}_{j}+\overline{\mathcal{E}}^{N_{f}}_{j}]$ & Interaction of fluctuations in $N_{m}$ and $N_{f}$\\

        $\overline{\mathcal{E}}^{w_{jm},w_{kf}}_{j}$ & $ \overline{r_{j}}(\overline{N_{m}}, \overline{N_{f}}, w_{jm}, w_{kf})- [\mathcal{E}^{0}_{j} +\overline{\mathcal{E}}^{w_{jm}}_j+\overline{\mathcal{E}}^{w_{kf}}_{j}]$ & Interaction of fluctuations in $w_{jm}$ and $w_{kf}$ \\

        $\overline{\mathcal{E}}^{N_{m}w_{jm}}_{j}$ & $\overline{r_{j}}(N_{m}, \overline{N_{f}}, w_{jm}, \overline{w_{kf}})- [\mathcal{E}^{0}_{j} +\overline{\mathcal{E}}^{N_{m}}_j+\overline{\mathcal{E}}^{w_{jm}}_{j}]$  & Interaction of fluctuations in $N_{m}$ and $w_{jm}$ \\


        $\overline{\mathcal{E}}^{N_{m}w_{kf}}_{j}$& $ \overline{r_{j}}(N_{m}, \overline{N_{f}}, \overline{w_{jm}}, w_{kf})- [\mathcal{E}^{0}_{j} +\overline{\mathcal{E}}^{N_{m}}_j+\overline{\mathcal{E}}^{w_{kf}}_{j}]$ & Interaction of fluctuations in $N_{m}$ and $w_{kf}$\\


        $\overline{\mathcal{E}}^{N_{f}w_{jm}}_{j}$& $\overline{r_{j}}(\overline{N_{m}}, N_{f}, w_{jm}, \overline{w_{kf}})- [\mathcal{E}^{0}_{j} +\overline{\mathcal{E}}^{N_{f}}_j+\overline{\mathcal{E}}^{w_{jm}}_{j}]$ & Interaction of variation in $N_{f}$ and $w_{jm}$ \\

        $\overline{\mathcal{E}}^{N_{f}w_{fk}}_{j}$& $ \overline{r_{j}}(\overline{N_{m}}, N_{f}, \overline{w_{jm}}, w_{kf})- [\mathcal{E}^{0}_{j} +\overline{\mathcal{E}}^{N_{f}}_j+\overline{\mathcal{E}}^{w_{kf}}_{j}]$ & Interaction of fluctuations $N_{f}$ and $w_{kf}$ \\

 %%%%%%%%%triplicate terms

        $\overline{\mathcal{E}}^{N_{m}w_{jm}w_{fk}}_{j}$& $ \overline{r_{j}}(N_{m}, \overline{N_{f}}, w_{jm}, w_{kf})- [\mathcal{E}^{0}_{j} +\overline{\mathcal{E}}^{N_{m}}_{j}+\overline{\mathcal{E}}^{w_{jm}}_j+\overline{\mathcal{E}}^{w_{kf}}_{j}]$  & Interaction of fluctuations in $N_{m}$, $w_{jm}$, and $w_{kf}$ \\

      $\overline{\mathcal{E}}^{N_{f}w_{jm}w_{fk}}_{j}$& $ \overline{r_{j}}(\overline{N_{m}}, N_{f}, w_{jm}, w_{kf})- [\mathcal{E}^{0}_{j} +\overline{\mathcal{E}}^{N_{f}}_{j}+\overline{\mathcal{E}}^{w_{jm}}_j+\overline{\mathcal{E}}^{w_{kf}}_{j}]$ & Interaction of fluctuations in $N_{f}$, $w_{jm}$, and $w_{kf}$ \\

      $\overline{\mathcal{E}}^{N_{m}N_{f}w_{jm}}_{j}$& $ \overline{r_{j}}(N_{m}, N_{f}, w_{jm}, \overline{w_{kf}})- [\mathcal{E}^{0}_{j} +\overline{\mathcal{E}}^{N_{m}}_{j}+\overline{\mathcal{E}}^{N_{f}}_{j}+\overline{\mathcal{E}}^{w_{jm}}_j]$ & Interaction of variation in $N_{m}$, $N_{f}$, and $w_{jm}$ \\



      $\overline{\mathcal{E}}^{N_{m}N_{f}w_{fk}}_{j}$& $ \overline{r_{j}}(N_{m}, N_{f}, \overline{w_{jm}}, w_{kf})- [\mathcal{E}^{0}_{j} +\overline{\mathcal{E}}^{N_{m}}_{j}+\overline{\mathcal{E}}^{N_{f}}_{j}+\overline{\mathcal{E}}^{w_{kf}}_j]$ & Interaction of fluctuations in $N_{m}$, $N_{f}$, and $w_{kf}$ \\

%%%%%%%last part
$\overline{\mathcal{E}}^{N_{m}N_{f}w_{jm}w_{fk}}_{j}$&  $ \overline{r_{j}}(N_{m}, N_{f}, w_{jm}, w_{kf})- [\mathcal{E}^{0}_{j} +\overline{\mathcal{E}}^{N_{m}}_{j}+\overline{\mathcal{E}}^{N_{f}}_{j}+\overline{\mathcal{E}}^{w_{jm}}_j+\overline{\mathcal{E}}^{w_{kf}}_j$] & Interaction of variation in $N_f$, $N_m$, $w_{jm}$, and $w_{kf}$ \\



         \bottomrule
    \end{tabular}}
    \label{tab:EllnerRs}
\end{table}



\begin{table}[h]
\fontsize{10}{18}\selectfont
\centering
\caption{This is a caption}
\begin{tabular}{@{}llll@{}}
\toprule
Parameter                    & Values                    & Description                                   &  \\ \midrule
$\sigma_{w}$ & 0.001, 0.1, 0.3, 0.5, 0.7, 0.9 & Effect size of fluctuations in fitness values &  \\
$\sigma_{g}$ & 0.001, 1, 10, 20, 30, 50 & Effect size of fluctuations in population sizes                                              &  \\
$\rho_{w}$  &  -0.75, 0, 0.75                         &   Correlation between fluctuations in fitness values                                            &  \\
$\rho_{g}$  &   -0.75, 0, 0.75                        &  Correlation between fluctuation in population sizes                                             &  \\ \bottomrule
\end{tabular}
\label{tab:fluctuations}
\end{table}



\begin{figure}[h]
  \centerline{\includegraphics[width=1\textwidth]{fluctuations.pdf}}
  \caption{ }
    \label{fig:fitness_deltas}
\end{figure}



\begin{figure}[h]
  \centerline{\includegraphics[width=1\textwidth]{fluctuations_two.pdf}}
  \caption{ }
    \label{fig:fitness_deltas}
\end{figure}



\clearpage
\bibliographystyle{ecology_letters}
\bibliography{coexistence.bib}

\end{document}
