\RequirePackage[]{lineno}
\documentclass[12pt]{article}
\usepackage{caption}
\usepackage{times}
\usepackage{setspace}
\usepackage{longtable}
\usepackage{amsmath}
\usepackage{booktabs}
\usepackage{float}
\usepackage{mathpazo}
\usepackage{times}
\usepackage{tikz}
\usepackage{graphicx}
\usepackage[hmargin=2.25cm, vmargin=2cm, headheight=15.5pt]{geometry}
\usepackage{multirow}
\usepackage{tcolorbox}
\usepackage{multicol}
\usepackage{tabularx}
\usepackage{rotating}
\usepackage{pdflscape}

\captionsetup[figure]{font=small}
\captionsetup[table]{font=small}

\usetikzlibrary{arrows,calc}
\geometry{margin=1in}

%\captionsetup{font=doublespacing, size= footnotesize}% Double-spaced float captions
\doublespacing
\DeclareCaptionJustification{double}{\DoubleSpacing}
% Reasonable page setup


\usepackage[]{natbib}
\bibpunct[; ]{(}{)}{;}{a}{,}{;}

% to avoid things being lost to overleaf comment bubbles
\long\def\authornote#1{%
    \leavevmode\unskip\raisebox{-3.5pt}{\rlap{$\scriptstyle\dagger$}}%
    \marginpar{\raggedright\hbadness=10000
        \def\baselinestretch{0.8}\tiny
        \it #1\par}}
\newcommand{\DBS}[1]{\authornote{DBS: #1}}
\newcommand{\ACL}[1]{\authornote{ACL: #1}}

\usepackage{authblk}
\renewcommand\Affilfont{\small}

\newenvironment{abox}[1]{
  \begin{tcolorbox}[float,title=#1, colback=blue!4]
  \fontsize{9}{10}\selectfont
  \begin{multicols}{2}
}{
  \end{multicols}
  \end{tcolorbox}
}


\newenvironment{ecolettcover}{\maketitle}{\clearpage}
\newenvironment{ecolettabstract}{\clearpage\section*{Abstract}}{\clearpage}
\tikzset{
	%Define standard arrow tip
	>=stealth',
	%Define style for different line styles
	help lines/.style={dashed, thick},
	axis/.style={<->},
	important line/.style={thick},
	connection/.style={thick, dotted},
}

%\title{The structural sensitivity of competition models: how model formulation changes our predictions of species coexistence}
\title{Coexistence of sexually antagonistic alleles}
\author[1]{Alba Cervantes-Loreto}
\author[1]{Michelle L.\ Marraffini}
\author[1]{Daniel B.\ Stouffer}
\author[1]{Sarah P.\ Flanagan}


\affil[1]{Centre for Integrative Ecology, School of Biological Sciences\\ University of Canterbury, Christchurch 8140, New Zealand}



% Include the date command, but leave its argument blank.
\date{}

%%%%%%%%%%%%%%%%% END OF PREAMBLE %%%%%%%%%%%%%%%%
\let\oldequation\equation
\let\oldendequation\endequation

\renewenvironment{equation}
  {\linenomathNonumbers\oldequation}
  {\oldendequation\endlinenomath}

% \pagestyle{empty}

\begin{document}
\linenumbers
% Double-space the manuscript.
\baselineskip30pt
\maketitle

\begin{ecolettcover}

%\centerline{{\sc Running Title:} The structural sensitivity of competition models}
\begin{center}
\begin{tabular}{ll}
\hline \\

\bf{Words in abstract}         & 203 \\
\bf{Words in manuscript}       & 5130\\
\bf{Number of references}      & 50  \\
\bf{Number of figures}			& 4 \\
\bf{Number of tables} 			& 2 \\
\bf{Number of text boxes}		& 0 \\
\bf{Corresponding author}      & Alba Cervantes-Loreto \\
\bf{Phone}                     & +64~369~2880 \\

\bf{Email}                     & alba.cervantesloreto@pg.canterbury.ac.nz \\
                                                                        \\
\hline
\end{tabular}
\end{center}

\maketitle

\end{ecolettcover}
\section{Abstract}



Sexually antagonistic selection (SAS) occurs when the selection in the traits or loci differs between the sexes. This sexual conflict offers the opportunity for maintaining polymorphism in a population, but it often results in the eventual fixation of the fitter allele. However, the effects of SAS have generally been studied under strong simplifying assumptions, such as constant populations and homogeneous environments, which could considerably change the expected outcomes of SAS. Thus, in this study, we examined how fluctuations in selection and population sizes contributed to the coexistence of sexually antagonistic alleles by adopting an ecological framework that allowed us to examine evolutionary dynamics through the same lens as the coexistence of competing species. We performed simulations of alleles invading a population while allowing selection and populations sizes to fluctuate over time.  Then, we quantified coexistence outcomes and the relative contribution of each type of fluctuation to each alleles' invasion growth rate. Our results showed that environmental fluctuations can dramatically increase the expected genetic variation under SAS. The positive contribution of fluctuations, however, depended on the sex and allele where invasion occurred. This study contributes to the growing body of work that shows the importance of non-constant environments on the maintenance of genetic diversity.

\section{Introduction}
The question of how genetic variation is maintained despite the effects of selection and drift is central within evolutionary biology \citep{walsh_evolution_2018}. Classical explanations include overdominance (heterozygote advantage) or frequency-dependent selection \citep{hedrick2007balancing}, but in the modern era of genomic data, all patterns of variation that exceed the expected variation under neutrality tend to be categorized broadly as balancing selection, regardless of the evolutionary mechanism \citep{mitchell-olds_which_2007}. In species with separate sexes, balancing selection can arise due to sexually antagonistic selection \citep{connallon2014balancing}, which occurs when the direction of natural selection on traits or loci differs between the sexes \citep{lande1980sexual,arnqvist2013sexual}.


Sexually antagonistic selection can maintain polymorphisms of otherwise disadvantageous alleles in a population \citep{gavrilets2014sexual}, which in turn can result in phenotypically distinct sexes that express different morphological, physiological, and behavioral traits \citep{mori2017sexual,connallon2018environmental}. Nonetheless,
the extent to which sexually antagonistic selection can maintain polymorphism in a population is thought to be limited \citep{connallon2012general}. This is because theoretical studies have found that the necessary parameter conditions that give rise to balancing selection are often highly restrictive \citep{kidwell1977regions,pamilo1979genic,hedrick1999antagonistic,curtsinger1994antagonistic, patten2010fitness, jordan2012potential}. Importantly, the effect of sexually antagonistic selection generally has been studied under strong simplifying assumptions such as constant population sizes and homogeneous environments  \citep{kidwell1977regions, pamilo1979genic, immler2012ploidally, jordan2012potential}. Studies that have explored the effect of sexually antagonistic selection with more realistic assumptions, such as temporal fluctuations in selection \citep{connallon_evolutionary_2018} or demographic fluctuations \citep{connallon2012general} have found that polymorphism can be maintained in a much wider set of conditions than classical studies predict. These results suggest that environmental fluctuations are essential to fully understand the effects of sexually antagonistic selection.

The contribution of environmental fluctuations to genetic diversity remains a debated issue in evolutionary biology. Classic theoretical models predict that temporal fluctuations in environmental conditions are unlikely to maintain a genetic polymorphism in haploid populations \citep{dempster1955maintenance,hedrick1974genetic,hedrick1986genetic}. However, other studies have found that fluctuating selection can maintain genetic variance when populations experience density dependence \citep{dean2005protecting}, on sex-linked traits \citep{reinhold2000maintenance}, or in populations where generations overlap \citep{ellner1994role, ellner1996patterns}. Similarly, temporal changes in population sizes have been shown to mitigate the effect of genetic drift in small populations \citep{pemberton1996maintenance} and in annual plant systems \citep{nunney2002effective}. Importantly, progress requires more than just identifying if environmental fluctuations can maintain genetic diversity in a population, but to quantify how exactly they contribute to its maintenance \citep{ellner2016quantify}.

Temporal variability in the environment has been shown to promote diversity maintenance in ecological contexts \citep{levins1979coexistence,armstrong1980competitive,chesson2000general,barabas_chessons_2018}. Note that from an ecological perspective, polymorphism of sexually antagonistic alleles is equivalent to the coexistence of species, and the fixation of either one of the alleles in a population is equivalent to competitive exclusion. Allelic polymorphism, thus, can be examined through the same lens as the coexistence of competing species. \citep{ellner1994role,ellner1996patterns,dean2005protecting,schreiber2010interactive}. The benefit of analyzing evolutionary dynamics through this lens is that the main theoretical framework used to examine how competing species coexist, often called Modern Coexistence Theory \citep{Chesson2000,chesson1994multispecies, barabas_chessons_2018}, allows the quantification of how environmental fluctuations contribute to coexistence. Despite that the use of Modern Coexistence Theory often requires complex mathematical analysis of the models describing the systems dynamics and restrictive assumptions to make them tractable \citep{barabas_chessons_2018}, recent computation approaches allow the quantification of the relative importance of environmental fluctuations to coexistence using simulations \citep{ellner2016quantify,ellner_expanded_2019,shoemaker2020}.

Here, we seek to explicitly quantify how temporal environmental fluctuations contribute to the maintenance of polymorphism under sexually antagonistic selection by using simulations.  We examined how fluctuations in selection values, fluctuations in population sizes, and their interactions can further or hinder polymorphism. In particular, we examined i) Can fluctuations in population sizes and selection values allow sexually antagonistic alleles to coexist when differences in their fitness would typically not allow them to? and ii) What are the relative contributions of different types of fluctuations that allow two sexually antagonistic alleles to be maintained in a population? Our study provides the tools to analyze sexual antagonism from a novel perspective and contributes to answering long-lasting questions regarding the effect of non-constant environments on genetic diversity.


\section{Methods}

We first present  a model that describes the evolutionary dynamics of sexually antagonistic alleles. We then show how we simulated different scenarios of alleles invading a population, where we allowed population sizes, selection, both, or neither to vary. Finally, we detail how we examined the relative contribution of each type of fluctuation to the maintenance of polymorphism.

\subsection*{Population dynamics of sexually antagonistic alleles}

 Our model examined evolution at a single, biallelic locus.  We examined the dynammics of two  sexually antagonistic alleles, $j$ and $k$, that affect fitness in the haploid state. The frequencies of each allele in each sex at the beginning of a life-cycle at time $t$ are given by:
 \begin{equation}
     p_{jm,t}= \frac{n_{jm,t}}{N_{m,t}}
     \label{first_pop}
 \end{equation}
 \begin{equation}
     p_{jf,t}= \frac{n_{jf,t}}{N_{f,t}}
 \end{equation}
 \begin{equation}
     p_{km,t}= \frac{N_{m,t}-n_{jm,t}}{N_{m,t}}
 \end{equation}
 \begin{equation}
     p_{kf,t}= \frac{N_{f,t}-n_{jf,t}}{N_{f,t}}
 \end{equation}
 where $N_{m,t}$ and $N_{f,t}$ are the total numbers of males and females in the population at time $t$, $n_{jf,t}$ is the number of females $f$ with allele $j$, and $n_{jm,t}$ is the number of males $m$ with allele $j$ at time $t$, respectively.

 The individuals in the population mate at random before selection occurs, and therefore the frequency of offspring with allele $j$ after mating, $p'_{j,t}$ can be expressed as:
 \begin{equation}
    p'_{j,t}= \frac{n_{jf}}{N_{f}} \frac{n_{jm}}{N_{m}} + \frac{1}{2} \frac{n_{jf}}{N_{f}} \frac{(N_{m}-n_{jm})}{N_{m}} +\frac{1}{2}
    \frac{(N_{f}-n_{jf})}{N_{f}} \frac{n_jm}{N_{m}} \,,
 \end{equation}
which upon rearranging and simplifying gives:
 \begin{equation}
    p'_{j,t}= \frac{N_{m,t}n_{jf,t}+ N_{f,t}n_{jm,t}}{2 N_{f}N_{m}} \,.
    \label{pprime}
 \end{equation}
 Selection acts upon these offspring in order to determine the allelic frequencies in females and males in the next generation, $t+1$. As an example, the frequency  of females with allele $j$ after selection is given by:
 \begin{equation}
    p_{jf, t+1}= \frac{n_{jf, t+1}}{N_{f,t+1}} = \frac{p'_{j,t}w_{jf}}{p'_{t,j}w_{jf}+ (1-p'_{t,j})w_{kf}}
    \label{next_gen}
 \end{equation}

The logarithmic per capita growth rate of allele $j$ in females is therefore given by the number of females carrying allele $j$ after selection divided by the original number of females carrying allele $j$:

 \begin{equation}
     r_{jf,t} = \ln \left( \frac{n_{jf, t+1}}{n_{jf,t}} \right)
     \label{canonical}
 \end{equation}

 An equivalent expression for the logarithmic per capita growth rate of allele $j$ in males $m$ can be obtained by exchanging $f$ for $m$ across the various subscripts in Eqn.~\ref{next_gen}.

 Polymorphism in a sexual population, however, is ultimately influenced by growth and establishment of an allele across both sexes. Therefore, the growth rate of allele $j$ across the entire population of females \emph{and} males is given by:
 \begin{equation}
     r_{j,t} = \ln \left( \frac{n_{jf, t+1} + n_{jm, t+1} }{n_{jf,t} + n_{jf,t} }  \right)
     \label{full}
 \end{equation}
An equivalent expression describes $r_{k,t}$, the growth rate of allele $k$.


Our model further assumed allele $j$ always has a high fitness in females ($w_{jf} = 1$) but variable fitness in males ($w_{jm} < 1$); and allele $k$ has a high fitness in males ($w_{km} = 1$)  but variable fitness in females ($w_{kf} < 1 $). The selection against allele $j$ in males is therefore $S_{m}= 1 - w_{jm}$, and the selection against allele $k$ in females is $S_{f}= 1 - w_{kf}$. When population sizes and selection values are constant,
selection mantains both alleles in the population, under the condition that:

\begin{equation}
\frac{S_{m}}{1+S_{m}} < S_{f} < \frac{S_{m}}{1-S_{m}}
\label{selection}
\end{equation}
\citep{kidwell1977regions,pamilo1979genic,patten2010fitness,connallon_evolutionary_2018}. Thus, the maintenance of polymorphism of sexually antagonistic alleles is solely determined by the values of $S_{m}$ and $S_{f}$. Note that in our model, the values $S_{m}$ and $S_{f}$ are bounded from $0$ to $1$. Therefore the parameter space of sexually antagonistic selection is within the range $ 0< S_{m}, S_{f} < 1$. Classic theoretical models predict that, in constant environments, polymorphism is maintained in $\approx 0.38$ of the parameter space \citep{kidwell1977regions,pamilo1979genic,connallon_evolutionary_2018}. Nonetheless it is unrealistic to assume population sizes and selection are constat through time. Temporal changes in population densities are ubiquitous in nature \citep{connallon2012general,reinhold2000maintenance}. Similarly, the effect of sexual selection has been show to vary through space and time \citep{kasumovic2008spatial}. If fluctuations in population sizes or selection values have an effect on the coexistence of sexually antagonistic alleles, it would be reflected in increases or decreases of the proportion of the parameter space of selection where polymorphism is maintained.







\clearpage
\bibliographystyle{ecology_letters}
\bibliography{coexistence.bib}

\end{document}
